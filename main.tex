\documentclass[10pt, aspectratio=1610]{beamer}

% Theming
\usetheme{Madrid}
\usecolortheme{beaver}

\usepackage{amsmath,amssymb,amsthm,mathtools}
\usepackage{algorithm2e, setspace}
\mathtoolsset{showonlyrefs}

\DeclareMathOperator*{\argmin}{arg\,min}
\DeclareMathOperator*{\argmax}{arg\,max}
\newcommand{\given}{\,|\,}
\newcommand{\dd}{\mathrm{d}}
\newtheorem{proposition}[theorem]{Proposition}

\usepackage{times}
\usepackage{tikz}
\usepackage{pgfplots}
\pgfplotsset{compat=newest}

% Natbib for citations.
\usepackage{natbib}
\bibliographystyle{unsrtnat}

\title{Policy Optimization with Markovian Score Climbing}
\date{February 16, 2024}
\author{Sahel Iqbal}
\institute{Aalto University, Finland}

\begin{document}
  \maketitle

  \AtBeginSection[] {
    \begin{frame}{Table of Contents}
		\tableofcontents[currentsection]
	\end{frame}
  }
  
  \begin{frame}{Works covered in this talk}
    \begin{enumerate}
      \item \emph{Risk-Sensitive Stochastic Optimal Control as Rao-Blackwellized Markovian Score Climbing}. Hany Abdulsamad, Sahel Iqbal, Adrien Corenflos, Simo S\"arkk\"a. 2023. arXiv preprint arXiv:2312.14000.
      \item \emph{Nesting Particle Filters for Experimental Design in Dynamical Systems}. Sahel Iqbal, Adrien Corenflos, Simo S\"arkk\"a, Hany Abdulsamad. 2024. arXiv preprint arXiv:2402.07868.
    \end{enumerate}
  \end{frame}

  \section{Stochastic Optimal Control as Inference}
    \begin{frame}{Expected Cost Criterion}
      Consider the discrete-time sequential decision making problem
      \begin{equation}\label{eq:soc_objective}
        \mathcal{J}(\pi) = \mathbb{E}_{x_{0:T}, u_{0:T}} \left[\sum_{t=0}^{T} c(x_t, u_t)\right],
      \end{equation}
      %
      where
      \begin{itemize}
        \item $x_t \in \mathbb{R}^m$ denote the states.
        \item $u_t \in \mathbb{R}^n$ denote the actions.
        \item $c(x_t, u_t) \geq 0$ is the stage cost.
      \end{itemize}
  
      \vskip 1cm
      The joint distribution of states and actions is
      \begin{equation}
          p(x_{0:T}, u_{0:T}) = \mu(x_{0}) \prod_{t=0}^{T-1} f(x_{t+1} \mid x_t, u_t) \, \prod_{t=0}^T \pi(u_t \mid x_t).
      \end{equation}
    \end{frame}

    \begin{frame}{Decision Making under Uncertainty}
        For a policy space $\Pi$, we want to find
        \begin{equation*}
            \pi^{*} = \argmin_{\pi \in \Pi} \medspace \mathcal{J}(\pi).
        \end{equation*}
        ~ \\
        For a parametric form $\pi(\cdot \given x_t) \equiv \pi_{\phi}(\cdot \given x_t)$ with $\phi \in \Phi \subseteq \mathbb{R}^{l}$
        \begin{equation*}
            \phi^{*} = \argmin_{\phi \in \Phi} \medspace \mathcal{J}(\phi).
        \end{equation*}
    \end{frame}

    \begin{frame}{The pseudo-likelihood}
      \begin{columns}
        \begin{column}{0.6\textwidth}
          \begin{itemize}
            \item Following \citet{toussaint2006probabilistic} and \citet{rawlik2013probabilistic}, let us introduce a binary random variable $y_t$ with likelihood
              \begin{equation}
                \smash{g(y_t = 1 \mid x_t, u_t) = \exp \big\{-\eta \, c(x_t, u_t)\big\}},
              \end{equation}
              %
              where $\eta \in \mathbb{R}_{>0}$.
            \item This can be interpreted as the \emph{probability of being optimal}.
            \item We can treat this as a pseudo-observation in a state-space model.
          \end{itemize}
        \end{column}
        \begin{column}{0.4\textwidth}
          \begin{figure}[htbp]
            \label{fig:dgm}
            \centering
            \tikzstyle{input}=[
    circle,
    very thick,
    minimum size=0.55cm,
    draw=black!80,
    fill=white!20
]

\tikzstyle{output}=[
    circle,
    very thick,
    minimum size=0.55cm,
    draw=black!80,
    fill=gray!20
]

\begin{tikzpicture}[>=latex,text height=0.8ex,text depth=0.25ex]
	\matrix[row sep=0.35cm,column sep=0.45cm] {
		& % Y LAYER
		\node (y_0) [output]{\large $y_0$}; &   &
		\node (y_t) [output]{\large $y_t$}; &   &
		\node (y_T) [output]{\large $y_T$}; &   &
		\\ & % X LAYER
		\node (z_0) [input]{\large $z_0$}; &   &
		\node (z_t) [input]{\large $z_t$}; &   &
		\node (z_T) [input]{\large $z_T$}; &   &
		\\
        & \node (theta) [input]{\large $\phi$}; & \\
	};
	\path[->]
	% Horizontal connections
	(z_0) edge[very thick, dashed] (z_t)
	(z_t) edge[very thick, dashed] (z_T)

	% Connections from x to y
	(z_0) edge[very thick] (y_0)
	(z_t) edge[very thick] (y_t)
	(z_T) edge[very thick] (y_T)

  % Connections from pi to u
	(theta) edge[very thick] (z_0)
	(theta) edge[bend right=20, very thick] (z_t)
	(theta) edge[bend right=20, very thick] (z_T)
	;
\end{tikzpicture}

          \end{figure}
        \end{column}
      \end{columns}
    \end{frame}

    \begin{frame}{Maximum Likelihood}
      Maximizing the marginal likelihood is equivalent to minimizing a risk-sensitive objective
      \begin{align}\label{eq:max_likelihood}
        \argmax_{\phi \in \Phi} \medspace \log p(y_{0:T} = 1 \mid \phi) &= \argmin_{\phi \in \Phi} \medspace -\frac{1}{\eta}\log \mathbb{E}_{p_{\phi}} \left[ \exp \big\{-\eta \textstyle \sum_{t=0}^{T} c(x_{t}, u_{t}) \big\} \right] \\
        &\coloneq \argmin_{\phi \in \Phi} \mathcal{J}_\eta(\phi)
      \end{align}
      %
      where $p_{\phi} \coloneqq p(x_{0:T}, u_{0:T} \mid \phi)$.

      \vskip 0.5cm
      Taylor expansion around $\eta = 0$ yields
      \begin{equation}
        \mathcal{J}_\eta(\phi) = \mathbb{E} \Bigg[ \sum_{t=0}^{T} c_{t}(x_{t}, u_{t}) \Bigg] - \frac{\eta}{2} \, \mathbb{V} \Bigg[ \sum_{t=0}^{T} c_{t}(x_{t}, u_{t}) \Bigg] + \mathcal{O}(\eta^{2}).
      \end{equation}
      %
      Thus, this log-marginal objective amounts to a risk-sensitive variant of the control objective modulated by the temperature $\eta$.
    \end{frame}

    \begin{frame}{Cleaning up the notation}
      \begin{itemize}
        \item Let us define an augmented state
          \begin{equation}
            z_t \coloneq (x_t, u_t)
          \end{equation}
        \item Additionally, let us define
          \begin{equation}
            \mu_{\phi}(z_0) \coloneqq \mu(x_0) \, \pi_{\phi}(u_0 \mid x_0), \quad f_{\phi}(z_{t+1} \mid z_t) \coloneqq f(x_{t+1} \mid x_t, u_t) \, \pi_{\phi}(u_{t+1} \mid x_{t+1}).
          \end{equation}
        \item We can then write a joint density
          \begin{equation}\label{eq:paper_1_ssm}
            p_{\phi}(z_{0:T}, y_{0:T}) = \mu_{\phi}(z_0) \prod_{t=0}^{T-1} f_{\phi}(z_{t+1} \mid z_t) \prod_{t=0}^T g(y_t \mid z_t).
          \end{equation}
        \item Our goal is to find the maximum likelihood estimate for the SSM specified by~\eqref{eq:paper_1_ssm}.
      \end{itemize}
    \end{frame}

    \begin{frame}{Complications}
      \begin{equation}
        p_{\phi}(z_{0:T}, y_{0:T}) = \mu_{\phi}(z_0) \prod_{t=0}^{T-1} f_{\phi}(z_{t+1} \mid z_t) \prod_{t=0}^T g(y_t \mid z_t).
      \end{equation}
      \begin{itemize}
        \item In general, the SSM can be
        \begin{itemize}
          \item non-linear,
          \item non-Gaussian,
          \item bounded support.
        \end{itemize}
      \item Gaussian approximations yield biased estimates of the marginal likelihood.
      \item The solution - particle methods!
      \end{itemize}
    \end{frame}

  \section{Markovian Score Climbing for MLE}
    \begin{frame}{Preliminaries - Particle filters}
      \begin{columns}
        \begin{column}{0.5\textwidth}
          A \emph{particle filter} is a sequential Monte Carlo algorithm that can be used to estimate
            \begin{itemize}
              \item the pathwise filtering distributions $p_\phi(x_{0:t} \mid y_{0:t})$,
              \item and the marginal likelihoods $p_\phi(y_{0:t})$.
          \end{itemize}
        \end{column}
        \begin{column}{0.5\textwidth}
          \begin{algorithm}[H]
            \setstretch{1.35}
            \SetAlgoLined
            \LinesNumbered
            \DontPrintSemicolon
            \For{$n \gets 1, \dots, N$}{
              Sample $z_0^n \sim \mu_\phi(\cdot)$ and $w_0^n = g(y_0 \given z_0^n)$. \;
              }
              \For{$t \gets 1, \dots, T$}{
                \For{$n \gets 1, \dots, N$}{
                  \tcp{Resample}
                  Sample $A_t^n$ with $\mathbb{P}(A_t^n = k) \propto w_{t-1}^k$. \;
                  \tcp{Propagate}
                  Sample $z_t^n \sim f_\phi(\cdot \given z_{t-1}^{A_t^n})$. \;
                  \tcp{Reweight}
                  Set $w_t^n = g(y_t \given z_t^n)$.\;
                }
              }
              \label{alg:particle_filter}
              \caption{Bootstrap particle filter.}
          \end{algorithm}
        \end{column}
      \end{columns}
    \end{frame}

    \begin{frame}{Preliminaries - Particle filters}
      \begin{itemize}
        \item Particle filters can provide unbiased estimates of the marginal likelihood for general SSMs, but obtaining its gradient is usually not possible.
        \item In literature this difficulty is often circumvented using Fisher's identity:
          \begin{equation}
            \nabla_\phi \, \ell(\phi) \coloneq \nabla_\phi \, p_\phi(y_{0:T} = 1) = \int \nabla_\phi \log p_{\phi}(z_{0:T}, y_{0:T}) \, p_\phi(z_{0:T} \given y_{0:T}) \, \dd z_{0:T}.
          \end{equation}
        \item The above integral can be estimated using samples from the smoothing distribution.
      \end{itemize} 
    \end{frame}

    \begin{frame}{Preliminaries - Backward sampling}
      \begin{columns}
        \begin{column}{0.5\textwidth}
          \begin{itemize}
            \item How do we obtain samples distributed according to $p_\phi(z_{0:T} \given y_{0:T})$?
            \item Algorithms~\ref{alg:particle_filter} and~\ref{alg:backward_sampling} together yield the \emph{forward filtering backward smoothing} algorithm, a kind of particle smoother.
          \end{itemize}          
        \end{column}
        \begin{column}{0.5\textwidth}
          \begin{algorithm}[H]
            \setstretch{1.35}
            \SetAlgoLined
            \LinesNumbered
            \DontPrintSemicolon
            Sample $B_T$ with $\mathbb{P}(B_T = k) \propto w_T^k$ \; 
            Set $z_T^* = z_T^{B_T}$ \;
            \For{$t \gets T-1, \dots, 0$}{
              Sample $B_t$ with $\mathbb{P}(B_t = k) \propto w_t^k \, f_\phi(z_{t+1}^{*} \given z_t^k)$ \; 
              Set $z_t^* = z_t^{B_t}$ \; 
            }
            \label{alg:backward_sampling}
            \caption{Backward sampling.}
          \end{algorithm}
        \end{column}
      \end{columns}
    \end{frame}

    \begin{frame}{Preliminaries - Gradient ascent using particle smoothing}
      We now have the building blocks in place to perform MLE through a gradient ascent procedure:
      \begin{equation}
          \phi_k = \phi_{k-1} + \gamma_k \, \widehat{\nabla_\phi \ell}(\phi_{k-1}),
      \end{equation}
      where
      \begin{itemize}
        \item $\widehat{\nabla_\phi \ell}(\phi_{k-1})$ is our stochastic estimate of the score function.
        \item $\gamma_k$ are step sizes that satisfy $\sum_{k=1}^\infty \gamma_k = \infty, \quad \sum_{k=1}^\infty \gamma_k^2 < \infty$.
      \end{itemize}

      \vspace{0.5cm}
      However, one more necessary criterion for the convergence of stochastic gradient ascent~\citep{robbins1951stochastic}, namely
      \begin{equation}
        \mathbb{E}\left[\widehat{\nabla_\phi \ell}(\phi_{k-1})\right] = \nabla_\phi \ell(\phi_{k-1})
      \end{equation}
    \end{frame}

\begin{frame}{Markovian Score Climbing For Likelihood Maximization}
        \begin{itemize}
            \setlength\itemsep{2em}
            \item Estimates of $\nabla_\phi \, \ell(\phi)$ via particle smoothing are biased for a finite number of particles $N$.
            \item We can do better by drawing samples from a Markov chain Monte Carlo kernel $\mathcal{K}$ ergodic w.r.t. $p_\phi(z_{0:T} \given y_{0:T})$, i.e.,
            \begin{equation*}
                Z^n \sim \mathcal{K}(\cdot \given \mathcal{K}(\cdot \given \ldots, \phi), \phi) := \mathcal{K}^n(\cdot \given z_{0:T}, \phi).
            \end{equation*}
            \item The $p_\phi(z_{0:T} \given y_{0:T})$-ergodic MCMC kernel $\mathcal{K}$ guarantees consistent estimates, no matter the number of samples $N$.
        \end{itemize}
    \end{frame}
\begin{frame}{Markovian Score Climbing For Likelihood Maximization}
        \begin{algorithm}[H]
            \setstretch{1.35}
            \SetAlgoLined
            \LinesNumbered
            \DontPrintSemicolon
            \KwIn{Initial trajectory $z^{0}_{0:T}$, initial parameters $\phi_0$, number of iterations $M$, step sizes $\gamma_{1:M}$, Markov kernel $\mathcal{K}$.}
            \KwOut{$\phi^* \approx \phi_{\text{MLE}}$}
            \For{$k \gets 1,\dots, M$}{
                Sample $z^{k}_{0:T} \sim \mathcal{K}(\cdot \given z^{k-1}_{0:T}, \phi_{k-1})$ \;
                Compute $\widehat{\nabla_\phi \ell}(\phi_{k-1}) \gets \nabla_\phi \log p_\phi(z^{k}_{0:T}, y_{0:T}) \vert_{\phi=\phi_{k-1}}$ \;
                Update $\phi_k \gets \phi_{k-1} + \gamma_k \widehat{\nabla_\phi \ell}(\phi_{k-1})$\;
            }
            \Return{$\phi_M$}
        \end{algorithm}
    \end{frame}

    \begin{frame}{\textcolor{blue}{Conditional} SMC with backward sampling}
        \begin{algorithm}[H]
            \setstretch{1.35}
            \SetAlgoLined
            \LinesNumbered
            \DontPrintSemicolon
            \KwIn{\textcolor{blue}{Reference trajectory $z_{0:T}$}}
            \tcp{Forward filtering}
            \textcolor{blue}{Set $z_0^1 = z_0$ and $w_0^1 = g(y_0 \given z_0)$} \;
            \For{$n \gets 2, \dots, N$}{
            Sample $z_0^n \sim \mu_\phi(\cdot)$ and $w_0^n = g(y_0 \given z_0^n)$ \;
            }
            \For{$t \gets 1, \dots, T$}{
                \textcolor{blue}{Set $z_t^1 = z_t$ and $w_t^1 = g(y_t \given z_t)$} \;
                \For{$n \gets 2, \dots, N$}{
                    Sample $A_t^n$ with $\mathbb{P}(A_t^n = k) \propto w_{t-1}^k$ \;
                    Sample $z_t^n \sim f_\phi(\cdot \given z_{t-1}^{A_t^n})$ and $w_t^n = g(y_t \given z_t^n)$\;
                }
            }
        \end{algorithm}
    \end{frame}

    \begin{frame}{Rao-Blackwellized Markovian Score Climbing}
        \begin{algorithm}[H]
            \setstretch{1.5}
            \SetAlgoLined
            \LinesNumbered
            \DontPrintSemicolon
            \KwIn{Trajectory $z_{0:T}$}
            \KwOut{Score estimate $\widehat{\nabla_\phi \ell}(\phi)$}
            Run forward filtering to obtain the filtered particles, their weights, and the resampling indices \;
            Backward-Sample $z^{n}_{0:T} \sim \mathcal{B}$ for all $n = 1, \dots, N$ \;
            Compute $\widehat{\nabla_\phi \ell}(\phi) \gets \frac{1}{N} \sum_{n=1}^N \nabla_\phi \log p_\phi(z^{n}_{0:T}, y_{0:T})$ \;
            \Return{$\widehat{\nabla_\phi \ell}(\phi)$}
        \end{algorithm}
    \end{frame}

    \begin{frame}{Numerical Validation}
      \begin{columns}
        \begin{column}{0.5\textwidth}
          \begin{figure}[htbp]
            \centering
            % This file was created with tikzplotlib v0.10.1.
\begin{tikzpicture}

\definecolor{crimson2143940}{RGB}{216,27,96}
\definecolor{darkgray176}{RGB}{176,176,176}
\definecolor{darkorange25512714}{RGB}{255,193,7}
\definecolor{forestgreen4416044}{RGB}{0,77,64}
\definecolor{steelblue31119180}{RGB}{30,136,229}

\begin{axis}[
    width=8cm,
    height=8cm,
    scaled y ticks=base 10:-3,
    tick pos=both,
    grid=both,
    minor tick num=3,
    try min ticks=6,
    grid style={line width=.1pt, draw=gray!10},
    major grid style={line width=.1pt, draw=gray!50},
    %xtick style={color=black},
    %ytick style={color=black},
    xmin=-1, xmax=21,
    ymin=1000.0, ymax=5000,
    legend style={
        nodes={scale=0.85, transform shape},
        at={(1,1)},
        anchor=north east
    },
    xlabel=Number of iterations,
    ylabel=Expected cost per trajectory,
    title=Pendulum,
]
\path [draw=steelblue31119180, semithick]
(axis cs:0,4213.08404695505)
--(axis cs:0,4383.76933169296);

\path [draw=steelblue31119180, semithick]
(axis cs:1,3775.56287002511)
--(axis cs:1,4119.62801186385);

\path [draw=steelblue31119180, semithick]
(axis cs:2,3022.29327301415)
--(axis cs:2,3446.61464531055);

\path [draw=steelblue31119180, semithick]
(axis cs:3,2517.23437258852)
--(axis cs:3,2780.80427615166);

\path [draw=steelblue31119180, semithick]
(axis cs:4,2299.88077296347)
--(axis cs:4,2525.94928031385);

\path [draw=steelblue31119180, semithick]
(axis cs:5,2138.32954347793)
--(axis cs:5,2492.62625290543);

\path [draw=steelblue31119180, semithick]
(axis cs:6,2159.27956421855)
--(axis cs:6,2423.00547662704);

\path [draw=steelblue31119180, semithick]
(axis cs:7,2174.94749330853)
--(axis cs:7,2386.14704406485);

\path [draw=steelblue31119180, semithick]
(axis cs:8,2225.43660736643)
--(axis cs:8,2419.84839028317);

\path [draw=steelblue31119180, semithick]
(axis cs:9,2143.64884788487)
--(axis cs:9,2453.15997325077);

\path [draw=steelblue31119180, semithick]
(axis cs:10,2159.56346284893)
--(axis cs:10,2470.14376272983);

\path [draw=steelblue31119180, semithick]
(axis cs:11,2178.90085494895)
--(axis cs:11,2477.54482120343);

\path [draw=steelblue31119180, semithick]
(axis cs:12,2203.94642704769)
--(axis cs:12,2457.00172031731);

\path [draw=steelblue31119180, semithick]
(axis cs:13,2211.1149699711)
--(axis cs:13,2450.3239529049);

\path [draw=steelblue31119180, semithick]
(axis cs:14,2195.72536981794)
--(axis cs:14,2490.61066381945);

\path [draw=steelblue31119180, semithick]
(axis cs:15,2179.68376250537)
--(axis cs:15,2512.96021049939);

\path [draw=steelblue31119180, semithick]
(axis cs:16,2195.01658585584)
--(axis cs:16,2533.06720303417);

\path [draw=steelblue31119180, semithick]
(axis cs:17,2102.74294389463)
--(axis cs:17,2565.08083586603);

\path [draw=steelblue31119180, semithick]
(axis cs:18,2102.54280553404)
--(axis cs:18,2553.16767707172);

\path [draw=steelblue31119180, semithick]
(axis cs:19,2067.6299288086)
--(axis cs:19,2543.60248912935);

\path [draw=steelblue31119180, semithick]
(axis cs:20,2081.18600684059)
--(axis cs:20,2550.47187916008);

\path [draw=darkorange25512714, semithick]
(axis cs:0,4239.98124787855)
--(axis cs:0,4340.69884211293);

\path [draw=darkorange25512714, semithick]
(axis cs:1,3826.74864695812)
--(axis cs:1,4203.06402355079);

\path [draw=darkorange25512714, semithick]
(axis cs:2,3015.36232947868)
--(axis cs:2,3580.4074509037);

\path [draw=darkorange25512714, semithick]
(axis cs:3,2654.28831213268)
--(axis cs:3,2904.00025846676);

\path [draw=darkorange25512714, semithick]
(axis cs:4,2355.77001486504)
--(axis cs:4,2697.88825899613);

\path [draw=darkorange25512714, semithick]
(axis cs:5,2214.33603652361)
--(axis cs:5,2641.8426610932);

\path [draw=darkorange25512714, semithick]
(axis cs:6,2217.03531670093)
--(axis cs:6,2649.98569125761);

\path [draw=darkorange25512714, semithick]
(axis cs:7,2226.57151783466)
--(axis cs:7,2635.95837111316);

\path [draw=darkorange25512714, semithick]
(axis cs:8,2204.78574666052)
--(axis cs:8,2647.32722715919);

\path [draw=darkorange25512714, semithick]
(axis cs:9,2231.70351369988)
--(axis cs:9,2610.70479530196);

\path [draw=darkorange25512714, semithick]
(axis cs:10,2168.8311869467)
--(axis cs:10,2572.56343515938);

\path [draw=darkorange25512714, semithick]
(axis cs:11,2097.74123188169)
--(axis cs:11,2555.62690127686);

\path [draw=darkorange25512714, semithick]
(axis cs:12,2019.13744825273)
--(axis cs:12,2576.63077460779);

\path [draw=darkorange25512714, semithick]
(axis cs:13,1963.09300703378)
--(axis cs:13,2546.50465193744);

\path [draw=darkorange25512714, semithick]
(axis cs:14,1894.90643406195)
--(axis cs:14,2536.74853758241);

\path [draw=darkorange25512714, semithick]
(axis cs:15,1821.52539871132)
--(axis cs:15,2557.36804466003);

\path [draw=darkorange25512714, semithick]
(axis cs:16,1775.22251477487)
--(axis cs:16,2548.30977574898);

\path [draw=darkorange25512714, semithick]
(axis cs:17,1691.72585215407)
--(axis cs:17,2560.67936098241);

\path [draw=darkorange25512714, semithick]
(axis cs:18,1680.54674997795)
--(axis cs:18,2550.83208590326);

\path [draw=darkorange25512714, semithick]
(axis cs:19,1635.20072117101)
--(axis cs:19,2574.32753796441);

\path [draw=darkorange25512714, semithick]
(axis cs:20,1603.30702924296)
--(axis cs:20,2593.7196781058);

\path [draw=forestgreen4416044, semithick]
(axis cs:0,3442.01572978692)
--(axis cs:0,4289.16582501954);

\path [draw=forestgreen4416044, semithick]
(axis cs:1,3156.99987110572)
--(axis cs:1,3758.39806642996);

\path [draw=forestgreen4416044, semithick]
(axis cs:2,2757.60363474025)
--(axis cs:2,4019.43206326942);

\path [draw=forestgreen4416044, semithick]
(axis cs:3,2788.22611800158)
--(axis cs:3,3515.35309762942);

\path [draw=forestgreen4416044, semithick]
(axis cs:4,2190.67683249995)
--(axis cs:4,3296.1889484115);

\path [draw=forestgreen4416044, semithick]
(axis cs:5,2138.21432312689)
--(axis cs:5,3084.55039735666);

\path [draw=forestgreen4416044, semithick]
(axis cs:6,2083.69040654514)
--(axis cs:6,3050.48580389767);

\path [draw=forestgreen4416044, semithick]
(axis cs:7,1973.00114445724)
--(axis cs:7,3206.83644304289);

\path [draw=forestgreen4416044, semithick]
(axis cs:8,2040.92135613399)
--(axis cs:8,3198.7055170687);

\path [draw=forestgreen4416044, semithick]
(axis cs:9,1944.92215987579)
--(axis cs:9,3018.36426528633);

\path [draw=forestgreen4416044, semithick]
(axis cs:10,1809.09504295236)
--(axis cs:10,2957.80856037678);

\path [draw=forestgreen4416044, semithick]
(axis cs:11,1584.94139443545)
--(axis cs:11,2744.26007916418);

\path [draw=forestgreen4416044, semithick]
(axis cs:12,1645.37521835516)
--(axis cs:12,2756.337980332);

\path [draw=forestgreen4416044, semithick]
(axis cs:13,1480.00841059511)
--(axis cs:13,2598.6146366905);

\path [draw=forestgreen4416044, semithick]
(axis cs:14,1275.46615209839)
--(axis cs:14,2362.84073254874);

\path [draw=forestgreen4416044, semithick]
(axis cs:15,1346.35172396132)
--(axis cs:15,2346.37625327005);

\path [draw=forestgreen4416044, semithick]
(axis cs:16,1297.93533328911)
--(axis cs:16,2185.10033150878);

\path [draw=forestgreen4416044, semithick]
(axis cs:17,1277.19002973573)
--(axis cs:17,2116.07544398307);

\path [draw=forestgreen4416044, semithick]
(axis cs:18,1266.56941132683)
--(axis cs:18,1892.39760237448);

\path [draw=forestgreen4416044, semithick]
(axis cs:19,1320.60228106896)
--(axis cs:19,1956.36435440566);

\path [draw=forestgreen4416044, semithick]
(axis cs:20,1292.33586897423)
--(axis cs:20,1849.15478148262);

\path [draw=crimson2143940, semithick]
(axis cs:0,3374.04422651336)
--(axis cs:0,4098.28744832855);

\path [draw=crimson2143940, semithick]
(axis cs:1,2494.19573712226)
--(axis cs:1,3778.82167269863);

\path [draw=crimson2143940, semithick]
(axis cs:2,2596.36304924003)
--(axis cs:2,2778.94166977038);

\path [draw=crimson2143940, semithick]
(axis cs:3,2361.00115644405)
--(axis cs:3,2746.03913560285);

\path [draw=crimson2143940, semithick]
(axis cs:4,2016.70231027374)
--(axis cs:4,2734.47842958715);

\path [draw=crimson2143940, semithick]
(axis cs:5,1788.2349421928)
--(axis cs:5,2837.20603846252);

\path [draw=crimson2143940, semithick]
(axis cs:6,1590.77579784608)
--(axis cs:6,2639.04962982006);

\path [draw=crimson2143940, semithick]
(axis cs:7,1321.95010567279)
--(axis cs:7,2261.33483333028);

\path [draw=crimson2143940, semithick]
(axis cs:8,1294.70739975606)
--(axis cs:8,2118.8970732481);

\path [draw=crimson2143940, semithick]
(axis cs:9,1149.41806636271)
--(axis cs:9,1579.40247124252);

\path [draw=crimson2143940, semithick]
(axis cs:10,1164.70885067031)
--(axis cs:10,1469.57796944479);

\path [draw=crimson2143940, semithick]
(axis cs:11,1174.13071050182)
--(axis cs:11,1331.37310082915);

\path [draw=crimson2143940, semithick]
(axis cs:12,1184.17889536996)
--(axis cs:12,1577.87530063994);

\path [draw=crimson2143940, semithick]
(axis cs:13,1205.0299803352)
--(axis cs:13,1480.78935522216);

\path [draw=crimson2143940, semithick]
(axis cs:14,1170.74086009759)
--(axis cs:14,1524.53701744321);

\path [draw=crimson2143940, semithick]
(axis cs:15,1167.12486776988)
--(axis cs:15,1626.89450171732);

\path [draw=crimson2143940, semithick]
(axis cs:16,1054.57178623361)
--(axis cs:16,1796.06619572116);

\path [draw=crimson2143940, semithick]
(axis cs:17,1157.60542369134)
--(axis cs:17,1228.52407835813);

\path [draw=crimson2143940, semithick]
(axis cs:18,1089.69272446918)
--(axis cs:18,1420.72100222113);

\path [draw=crimson2143940, semithick]
(axis cs:19,1160.48002628421)
--(axis cs:19,1292.62191616333);

\path [draw=crimson2143940, semithick]
(axis cs:20,1130.14499141678)
--(axis cs:20,1297.28869719049);

\addplot [semithick, steelblue31119180, mark=*, mark size=2, mark options={solid}]
table {%
0 4298.426689324
1 3947.59544094448
2 3234.45395916235
3 2649.01932437009
4 2412.91502663866
5 2315.47789819168
6 2291.1425204228
7 2280.54726868669
8 2322.6424988248
9 2298.40441056782
10 2314.85361278938
11 2328.22283807619
12 2330.4740736825
13 2330.719461438
14 2343.16801681869
15 2346.32198650238
16 2364.041894445
17 2333.91188988033
18 2327.85524130288
19 2305.61620896898
20 2315.82894300034
};
\addplot [semithick, darkorange25512714, mark=triangle, mark size=2, mark options={solid}]
table {%
0 4290.34004499574
1 4014.90633525445
2 3297.88489019119
3 2779.14428529972
4 2526.82913693059
5 2428.08934880841
6 2433.51050397927
7 2431.26494447391
8 2426.05648690985
9 2421.20415450092
10 2370.69731105304
11 2326.68406657927
12 2297.88411143026
13 2254.79882948561
14 2215.82748582218
15 2189.44672168567
16 2161.76614526193
17 2126.20260656824
18 2115.68941794061
19 2104.76412956771
20 2098.51335367438
};
\addplot [semithick, forestgreen4416044, mark=square, mark size=2, mark options={solid}]
table {%
0 3865.59077740323
1 3457.69896876784
2 3388.51784900483
3 3151.7896078155
4 2743.43289045573
5 2611.38236024177
6 2567.0881052214
7 2589.91879375007
8 2619.81343660135
9 2481.64321258106
10 2383.45180166457
11 2164.60073679982
12 2200.85659934358
13 2039.3115236428
14 1819.15344232357
15 1846.36398861569
16 1741.51783239895
17 1696.6327368594
18 1579.48350685065
19 1638.48331773731
20 1570.74532522843
};
\addplot [semithick, crimson2143940, mark=diamond, mark size=2, mark options={solid}]
table {%
0 3736.16583742096
1 3136.50870491045
2 2687.6523595052
3 2553.52014602345
4 2375.59036993045
5 2312.72049032766
6 2114.91271383307
7 1791.64246950153
8 1706.80223650208
9 1364.41026880261
10 1317.14341005755
11 1252.75190566548
12 1381.02709800495
13 1342.90966777868
14 1347.6389387704
15 1397.0096847436
16 1425.31899097738
17 1193.06475102473
18 1255.20686334516
19 1226.55097122377
20 1213.71684430363
};
\legend{PPO, TRPO, SMC, RB-CSMC}
\end{axis}

\end{tikzpicture}

          \end{figure}
        \end{column}
        \begin{column}{0.5\textwidth}
          \begin{figure}[htbp]
            \centering
            % This file was created with tikzplotlib v0.10.1.
\begin{tikzpicture}

\definecolor{crimson2143940}{RGB}{216,27,96}
\definecolor{darkgray176}{RGB}{176,176,176}
\definecolor{darkorange25512714}{RGB}{255,193,7}
\definecolor{forestgreen4416044}{RGB}{0,77,64}
\definecolor{steelblue31119180}{RGB}{30,136,229}

\begin{axis}[
    width=6cm,
    height=6cm,
    scaled y ticks=base 10:-3,
    tick pos=both,
    grid=both,
    minor tick num=3,
    try min ticks=6,
    grid style={line width=.1pt, draw=gray!10},
    major grid style={line width=.1pt, draw=gray!50},
    %xtick style={color=black},
    %ytick style={color=black},
    xmin=-1, xmax=21,
    ymin=1000, ymax=9000,
    legend style={
        nodes={scale=0.85, transform shape},
        at={(1,1)},
        anchor=north east
    },
    xlabel=Number of iterations,
    ylabel=Expected cost per trajectory,
    title=Cart-Pole,
]
\path [draw=steelblue31119180, semithick]
(axis cs:0,3514.59883405117)
--(axis cs:0,4292.48191913703);

\path [draw=steelblue31119180, semithick]
(axis cs:1,2856.16148572971)
--(axis cs:1,3298.18925469358);

\path [draw=steelblue31119180, semithick]
(axis cs:2,2557.9794151522)
--(axis cs:2,2672.09574269788);

\path [draw=steelblue31119180, semithick]
(axis cs:3,2415.62720876809)
--(axis cs:3,2546.65808905772);

\path [draw=steelblue31119180, semithick]
(axis cs:4,2366.65596365858)
--(axis cs:4,2548.60424634038);

\path [draw=steelblue31119180, semithick]
(axis cs:5,2427.04795264214)
--(axis cs:5,2719.24432191694);

\path [draw=steelblue31119180, semithick]
(axis cs:6,2457.01369302183)
--(axis cs:6,2807.91746193022);

\path [draw=steelblue31119180, semithick]
(axis cs:7,2482.76694051297)
--(axis cs:7,2848.10069135146);

\path [draw=steelblue31119180, semithick]
(axis cs:8,2513.37587352098)
--(axis cs:8,2935.05047909243);

\path [draw=steelblue31119180, semithick]
(axis cs:9,2458.71210860727)
--(axis cs:9,3016.66784094755);

\path [draw=steelblue31119180, semithick]
(axis cs:10,2449.85071445921)
--(axis cs:10,3157.58738699016);

\path [draw=steelblue31119180, semithick]
(axis cs:11,2486.34671274944)
--(axis cs:11,3199.98049231875);

\path [draw=steelblue31119180, semithick]
(axis cs:12,2443.91162558427)
--(axis cs:12,3162.42395653683);

\path [draw=steelblue31119180, semithick]
(axis cs:13,2441.72308714191)
--(axis cs:13,3300.72298415089);

\path [draw=steelblue31119180, semithick]
(axis cs:14,2352.70554629693)
--(axis cs:14,3385.33542101324);

\path [draw=steelblue31119180, semithick]
(axis cs:15,2404.66412054652)
--(axis cs:15,3260.18181781051);

\path [draw=steelblue31119180, semithick]
(axis cs:16,2430.88316854844)
--(axis cs:16,3317.3816868091);

\path [draw=steelblue31119180, semithick]
(axis cs:17,2407.84979804146)
--(axis cs:17,3327.45761345584);

\path [draw=steelblue31119180, semithick]
(axis cs:18,2431.84925815291)
--(axis cs:18,3316.24252705514);

\path [draw=steelblue31119180, semithick]
(axis cs:19,2421.61828318772)
--(axis cs:19,3259.94380817981);

\path [draw=steelblue31119180, semithick]
(axis cs:20,2414.01562320355)
--(axis cs:20,3285.67022857181);

\path [draw=darkorange25512714, semithick]
(axis cs:0,3595.67863644136)
--(axis cs:0,4161.08304079065);

\path [draw=darkorange25512714, semithick]
(axis cs:1,3325.65233124313)
--(axis cs:1,3733.03959686048);

\path [draw=darkorange25512714, semithick]
(axis cs:2,2787.21592614498)
--(axis cs:2,3217.057035931);

\path [draw=darkorange25512714, semithick]
(axis cs:3,2427.0623926601)
--(axis cs:3,3139.59168518089);

\path [draw=darkorange25512714, semithick]
(axis cs:4,2376.92313906172)
--(axis cs:4,2820.94729987976);

\path [draw=darkorange25512714, semithick]
(axis cs:5,2305.34504119129)
--(axis cs:5,2672.9300003075);

\path [draw=darkorange25512714, semithick]
(axis cs:6,2261.14019556055)
--(axis cs:6,2588.28420338697);

\path [draw=darkorange25512714, semithick]
(axis cs:7,2248.40612127092)
--(axis cs:7,2623.08930437373);

\path [draw=darkorange25512714, semithick]
(axis cs:8,2191.28666196223)
--(axis cs:8,2749.23314605295);

\path [draw=darkorange25512714, semithick]
(axis cs:9,2109.97967648965)
--(axis cs:9,2850.45686630929);

\path [draw=darkorange25512714, semithick]
(axis cs:10,2162.9101350743)
--(axis cs:10,2841.43213094229);

\path [draw=darkorange25512714, semithick]
(axis cs:11,2165.89935412759)
--(axis cs:11,2809.28287617367);

\path [draw=darkorange25512714, semithick]
(axis cs:12,2138.81605009428)
--(axis cs:12,2895.34106421726);

\path [draw=darkorange25512714, semithick]
(axis cs:13,2134.71651274591)
--(axis cs:13,2882.7258019441);

\path [draw=darkorange25512714, semithick]
(axis cs:14,2184.70860586695)
--(axis cs:14,2871.49304960635);

\path [draw=darkorange25512714, semithick]
(axis cs:15,2175.91911700133)
--(axis cs:15,2837.98662439978);

\path [draw=darkorange25512714, semithick]
(axis cs:16,2140.01961229716)
--(axis cs:16,2880.53007456611);

\path [draw=darkorange25512714, semithick]
(axis cs:17,2167.11551615577)
--(axis cs:17,2793.64615801525);

\path [draw=darkorange25512714, semithick]
(axis cs:18,2168.78338415491)
--(axis cs:18,2772.20997223446);

\path [draw=darkorange25512714, semithick]
(axis cs:19,2120.19688096146)
--(axis cs:19,2777.00589541093);

\path [draw=darkorange25512714, semithick]
(axis cs:20,2164.72757891378)
--(axis cs:20,2804.04238342474);

\path [draw=forestgreen4416044, semithick]
(axis cs:0,3286.44725207109)
--(axis cs:0,9854.01256483686);

\path [draw=forestgreen4416044, semithick]
(axis cs:1,3048.82372814697)
--(axis cs:1,5607.77322351749);

\path [draw=forestgreen4416044, semithick]
(axis cs:2,2637.64485233356)
--(axis cs:2,3155.66094493643);

\path [draw=forestgreen4416044, semithick]
(axis cs:3,2467.85475236433)
--(axis cs:3,2869.42322538535);

\path [draw=forestgreen4416044, semithick]
(axis cs:4,2361.84949623607)
--(axis cs:4,2825.42065168267);

\path [draw=forestgreen4416044, semithick]
(axis cs:5,2322.57882569076)
--(axis cs:5,2733.26657578471);

\path [draw=forestgreen4416044, semithick]
(axis cs:6,2154.69355889984)
--(axis cs:6,2619.18388610438);

\path [draw=forestgreen4416044, semithick]
(axis cs:7,2061.92009127385)
--(axis cs:7,2561.44874072853);

\path [draw=forestgreen4416044, semithick]
(axis cs:8,1782.7945459782)
--(axis cs:8,2767.4801816319);

\path [draw=forestgreen4416044, semithick]
(axis cs:9,1716.96710114668)
--(axis cs:9,2946.22642168012);

\path [draw=forestgreen4416044, semithick]
(axis cs:10,1831.24788914492)
--(axis cs:10,3095.59423766039);

\path [draw=forestgreen4416044, semithick]
(axis cs:11,1839.09081366049)
--(axis cs:11,3410.67963749115);

\path [draw=forestgreen4416044, semithick]
(axis cs:12,1943.81268087774)
--(axis cs:12,3679.51850243148);

\path [draw=forestgreen4416044, semithick]
(axis cs:13,1928.95492680777)
--(axis cs:13,3801.24701396931);

\path [draw=forestgreen4416044, semithick]
(axis cs:14,1857.49481909867)
--(axis cs:14,3362.40351006053);

\path [draw=forestgreen4416044, semithick]
(axis cs:15,1912.3039750197)
--(axis cs:15,2688.38372938353);

\path [draw=forestgreen4416044, semithick]
(axis cs:16,1827.05797758798)
--(axis cs:16,2560.77923486197);

\path [draw=forestgreen4416044, semithick]
(axis cs:17,1687.90461649693)
--(axis cs:17,2732.48758453913);

\path [draw=forestgreen4416044, semithick]
(axis cs:18,1695.81900008189)
--(axis cs:18,2870.13446978204);

\path [draw=forestgreen4416044, semithick]
(axis cs:19,1714.34292450914)
--(axis cs:19,2602.83840741546);

\path [draw=forestgreen4416044, semithick]
(axis cs:20,1601.17907284307)
--(axis cs:20,2476.20979477816);

\path [draw=crimson2143940, semithick]
(axis cs:0,2636.16752100694)
--(axis cs:0,14968.5458487787);

\path [draw=crimson2143940, semithick]
(axis cs:1,1787.0565909443)
--(axis cs:1,3184.75324542095);

\path [draw=crimson2143940, semithick]
(axis cs:2,1518.81098052665)
--(axis cs:2,2696.81540980844);

\path [draw=crimson2143940, semithick]
(axis cs:3,1358.16785344453)
--(axis cs:3,2637.29468922661);

\path [draw=crimson2143940, semithick]
(axis cs:4,1339.55021279615)
--(axis cs:4,2201.31896574572);

\path [draw=crimson2143940, semithick]
(axis cs:5,1417.0502965748)
--(axis cs:5,2304.88778839532);

\path [draw=crimson2143940, semithick]
(axis cs:6,1554.41752123425)
--(axis cs:6,2113.66560075351);

\path [draw=crimson2143940, semithick]
(axis cs:7,1420.16771625023)
--(axis cs:7,2383.63203659687);

\path [draw=crimson2143940, semithick]
(axis cs:8,1379.5683562811)
--(axis cs:8,2117.52128806283);

\path [draw=crimson2143940, semithick]
(axis cs:9,1497.9234482507)
--(axis cs:9,1869.10834360893);

\path [draw=crimson2143940, semithick]
(axis cs:10,1453.01365634792)
--(axis cs:10,1921.44482964988);

\path [draw=crimson2143940, semithick]
(axis cs:11,1424.88457860972)
--(axis cs:11,2027.43241560029);

\path [draw=crimson2143940, semithick]
(axis cs:12,1482.18689238059)
--(axis cs:12,2125.17289833442);

\path [draw=crimson2143940, semithick]
(axis cs:13,1479.33915140981)
--(axis cs:13,1894.48466841584);

\path [draw=crimson2143940, semithick]
(axis cs:14,1513.96679923407)
--(axis cs:14,1844.233416284);

\path [draw=crimson2143940, semithick]
(axis cs:15,1431.3873746189)
--(axis cs:15,1825.74186921492);

\path [draw=crimson2143940, semithick]
(axis cs:16,1446.24558571531)
--(axis cs:16,1848.36774342474);

\path [draw=crimson2143940, semithick]
(axis cs:17,1363.06193250651)
--(axis cs:17,1859.68161200067);

\path [draw=crimson2143940, semithick]
(axis cs:18,1482.09925442005)
--(axis cs:18,1759.3095356151);

\path [draw=crimson2143940, semithick]
(axis cs:19,1419.64919707007)
--(axis cs:19,1696.1577228295);

\path [draw=crimson2143940, semithick]
(axis cs:20,1372.05032353848)
--(axis cs:20,2063.50788850908);

\addplot [semithick, steelblue31119180, mark=*, mark size=2, mark options={solid}]
table {%
0 3903.5403765941
1 3077.17537021165
2 2615.03757892504
3 2481.14264891291
4 2457.63010499948
5 2573.14613727954
6 2632.46557747603
7 2665.43381593222
8 2724.21317630671
9 2737.68997477741
10 2803.71905072468
11 2843.1636025341
12 2803.16779106055
13 2871.2230356464
14 2869.02048365509
15 2832.42296917852
16 2874.13242767877
17 2867.65370574865
18 2874.04589260403
19 2840.78104568376
20 2849.84292588768
};
\addplot [semithick, darkorange25512714, mark=triangle, mark size=2, mark options={solid}]
table {%
0 3878.380838616
1 3529.3459640518
2 3002.13648103799
3 2783.3270389205
4 2598.93521947074
5 2489.1375207494
6 2424.71219947376
7 2435.74771282233
8 2470.25990400759
9 2480.21827139947
10 2502.1711330083
11 2487.59111515063
12 2517.07855715577
13 2508.72115734501
14 2528.10082773665
15 2506.95287070056
16 2510.27484343164
17 2480.38083708551
18 2470.49667819469
19 2448.60138818619
20 2484.38498116926
};
\addplot [semithick, forestgreen4416044, mark=square, mark size=2, mark options={solid}]
table {%
0 6570.22990845398
1 4328.29847583223
2 2896.65289863499
3 2668.63898887484
4 2593.63507395937
5 2527.92270073774
6 2386.93872250211
7 2311.68441600119
8 2275.13736380505
9 2331.5967614134
10 2463.42106340266
11 2624.88522557582
12 2811.66559165461
13 2865.10097038854
14 2609.9491645796
15 2300.34385220162
16 2193.91860622498
17 2210.19610051803
18 2282.97673493196
19 2158.5906659623
20 2038.69443381062
};
\addplot [semithick, crimson2143940, mark=diamond, mark size=2, mark options={solid}]
table {%
0 8802.3566848928
1 2485.90491818263
2 2107.81319516754
3 1997.73127133557
4 1770.43458927094
5 1860.96904248506
6 1834.04156099388
7 1901.89987642355
8 1748.54482217196
9 1683.51589592981
10 1687.2292429989
11 1726.158497105
12 1803.67989535751
13 1686.91190991282
14 1679.10010775903
15 1628.56462191691
16 1647.30666457002
17 1611.37177225359
18 1620.70439501757
19 1557.90345994979
20 1717.77910602378
};
\legend{PPO, TRPO, SMC, RB-CSMC}
\end{axis}

\end{tikzpicture}

          \end{figure}
        \end{column}
      \end{columns}
    \end{frame}

    \begin{frame}{Numerical Validation}
      \begin{columns}
        \begin{column}{0.5\textwidth}
          \begin{figure}[htbp]
            \centering
            % This file was created with tikzplotlib v0.10.1.
\begin{tikzpicture}

\definecolor{crimson2143940}{RGB}{216,27,96}
\definecolor{darkgray176}{RGB}{176,176,176}
\definecolor{darkorange25512714}{RGB}{255,193,7}
\definecolor{forestgreen4416044}{RGB}{0,77,64}
\definecolor{steelblue31119180}{RGB}{30,136,229}

\begin{axis}[
    width=6cm,
    height=6cm,
    scaled y ticks=base 10:-3,
    tick pos=both,
    grid=both,
    minor tick num=3,
    try min ticks=6,
    grid style={line width=.1pt, draw=gray!10},
    major grid style={line width=.1pt, draw=gray!50},
    %xtick style={color=black},
    %ytick style={color=black},
    xmin=-3, xmax=63,
    ymin=1000, ymax=12000,
    legend style={
        nodes={scale=0.85, transform shape},
        at={(1,1)},
        anchor=north east
    },
    xlabel=Number of iterations,
    ylabel=Expected cost per trajectory,
    title=Double Pendulum,
]
\path [draw=steelblue31119180, semithick]
(axis cs:0,9781.01825640702)
--(axis cs:0,10497.8200322552);

\path [draw=steelblue31119180, semithick]
(axis cs:3,7451.26845255326)
--(axis cs:3,8223.86715581045);

\path [draw=steelblue31119180, semithick]
(axis cs:6,5458.43297695593)
--(axis cs:6,7750.87776143246);

\path [draw=steelblue31119180, semithick]
(axis cs:9,5179.34861082324)
--(axis cs:9,7425.58533132673);

\path [draw=steelblue31119180, semithick]
(axis cs:12,4382.88877574496)
--(axis cs:12,6926.52338527657);

\path [draw=steelblue31119180, semithick]
(axis cs:15,3558.47506390363)
--(axis cs:15,5970.65571946054);

\path [draw=steelblue31119180, semithick]
(axis cs:18,2688.72096864883)
--(axis cs:18,4864.92274487825);

\path [draw=steelblue31119180, semithick]
(axis cs:21,2158.62140298183)
--(axis cs:21,4145.4602272953);

\path [draw=steelblue31119180, semithick]
(axis cs:24,2051.60194378141)
--(axis cs:24,3535.31454516272);

\path [draw=steelblue31119180, semithick]
(axis cs:27,1930.4861896838)
--(axis cs:27,3114.35380555735);

\path [draw=steelblue31119180, semithick]
(axis cs:30,1912.32675974637)
--(axis cs:30,3080.0375457669);

\path [draw=steelblue31119180, semithick]
(axis cs:33,1686.24222118566)
--(axis cs:33,3039.73405202958);

\path [draw=steelblue31119180, semithick]
(axis cs:36,1604.99233887748)
--(axis cs:36,3070.26109597667);

\path [draw=steelblue31119180, semithick]
(axis cs:39,1580.49303181514)
--(axis cs:39,2907.56131651246);

\path [draw=steelblue31119180, semithick]
(axis cs:42,1419.9228223128)
--(axis cs:42,3295.21846932361);

\path [draw=steelblue31119180, semithick]
(axis cs:45,1387.71298513582)
--(axis cs:45,3289.43608491951);

\path [draw=steelblue31119180, semithick]
(axis cs:48,1413.6514188377)
--(axis cs:48,3060.95647825368);

\path [draw=steelblue31119180, semithick]
(axis cs:51,1403.40053953831)
--(axis cs:51,2905.67632442785);

\path [draw=steelblue31119180, semithick]
(axis cs:54,1407.87147796582)
--(axis cs:54,2828.13779338046);

\path [draw=steelblue31119180, semithick]
(axis cs:57,1384.79171209435)
--(axis cs:57,2858.86837023816);

\path [draw=steelblue31119180, semithick]
(axis cs:60,1268.4896321138)
--(axis cs:60,3238.06914658854);

\path [draw=darkorange25512714, semithick]
(axis cs:0,9839.80314800701)
--(axis cs:0,10178.2139816316);

\path [draw=darkorange25512714, semithick]
(axis cs:3,8305.29703559156)
--(axis cs:3,9161.7989170679);

\path [draw=darkorange25512714, semithick]
(axis cs:6,5869.70515768519)
--(axis cs:6,8278.59003358874);

\path [draw=darkorange25512714, semithick]
(axis cs:9,4230.69773073299)
--(axis cs:9,7259.48733525169);

\path [draw=darkorange25512714, semithick]
(axis cs:12,3838.74087468201)
--(axis cs:12,6912.13012008653);

\path [draw=darkorange25512714, semithick]
(axis cs:15,3481.05880791934)
--(axis cs:15,6598.82948827936);

\path [draw=darkorange25512714, semithick]
(axis cs:18,3060.86469183338)
--(axis cs:18,6109.30339765108);

\path [draw=darkorange25512714, semithick]
(axis cs:21,2617.26285893903)
--(axis cs:21,5647.73959612429);

\path [draw=darkorange25512714, semithick]
(axis cs:24,2462.03680326748)
--(axis cs:24,5097.47525535224);

\path [draw=darkorange25512714, semithick]
(axis cs:27,2479.11990587921)
--(axis cs:27,4412.71736780727);

\path [draw=darkorange25512714, semithick]
(axis cs:30,2415.09110025343)
--(axis cs:30,4083.5218485265);

\path [draw=darkorange25512714, semithick]
(axis cs:33,2228.1612819246)
--(axis cs:33,3879.64661252475);

\path [draw=darkorange25512714, semithick]
(axis cs:36,2227.21837504513)
--(axis cs:36,3409.89605806769);

\path [draw=darkorange25512714, semithick]
(axis cs:39,2039.90199024983)
--(axis cs:39,3371.23155846448);

\path [draw=darkorange25512714, semithick]
(axis cs:42,1945.90122774608)
--(axis cs:42,3408.62674803467);

\path [draw=darkorange25512714, semithick]
(axis cs:45,2000.76006181235)
--(axis cs:45,3108.14223029226);

\path [draw=darkorange25512714, semithick]
(axis cs:48,1967.68491920454)
--(axis cs:48,2997.56542518049);

\path [draw=darkorange25512714, semithick]
(axis cs:51,1968.54770629822)
--(axis cs:51,2952.31264974986);

\path [draw=darkorange25512714, semithick]
(axis cs:54,1927.13576046316)
--(axis cs:54,2926.61007495138);

\path [draw=darkorange25512714, semithick]
(axis cs:57,1910.50324523432)
--(axis cs:57,2862.95880668025);

\path [draw=darkorange25512714, semithick]
(axis cs:60,1906.59898609545)
--(axis cs:60,2838.40992597584);

\path [draw=forestgreen4416044, semithick]
(axis cs:0,6142.52579850615)
--(axis cs:0,11893.4519281822);

\path [draw=forestgreen4416044, semithick]
(axis cs:3,4134.79499521663)
--(axis cs:3,7604.26134392331);

\path [draw=forestgreen4416044, semithick]
(axis cs:6,4027.30291885227)
--(axis cs:6,6450.76735457472);

\path [draw=forestgreen4416044, semithick]
(axis cs:9,3578.5300892024)
--(axis cs:9,4699.44862963747);

\path [draw=forestgreen4416044, semithick]
(axis cs:12,3206.81884789466)
--(axis cs:12,3894.53640766546);

\path [draw=forestgreen4416044, semithick]
(axis cs:15,3136.29471077372)
--(axis cs:15,4071.57722712485);

\path [draw=forestgreen4416044, semithick]
(axis cs:18,3153.48608037754)
--(axis cs:18,3797.89790841298);

\path [draw=forestgreen4416044, semithick]
(axis cs:21,2829.9703055808)
--(axis cs:21,4048.58921326533);

\path [draw=forestgreen4416044, semithick]
(axis cs:24,2974.54422262129)
--(axis cs:24,3377.47892701299);

\path [draw=forestgreen4416044, semithick]
(axis cs:27,2861.81129707118)
--(axis cs:27,3410.59729988973);

\path [draw=forestgreen4416044, semithick]
(axis cs:30,2786.05222318068)
--(axis cs:30,4112.58039610398);

\path [draw=forestgreen4416044, semithick]
(axis cs:33,2850.28545566731)
--(axis cs:33,3514.47556801468);

\path [draw=forestgreen4416044, semithick]
(axis cs:36,2848.07868710722)
--(axis cs:36,3165.34638798955);

\path [draw=forestgreen4416044, semithick]
(axis cs:39,2660.10558749268)
--(axis cs:39,3478.97251772312);

\path [draw=forestgreen4416044, semithick]
(axis cs:42,2831.41337241338)
--(axis cs:42,3295.6900410662);

\path [draw=forestgreen4416044, semithick]
(axis cs:45,2786.18678335787)
--(axis cs:45,3314.45155281853);

\path [draw=forestgreen4416044, semithick]
(axis cs:48,2751.93663916066)
--(axis cs:48,3271.36280197407);

\path [draw=forestgreen4416044, semithick]
(axis cs:51,2711.08056810518)
--(axis cs:51,3193.20280733772);

\path [draw=forestgreen4416044, semithick]
(axis cs:54,2813.16577436311)
--(axis cs:54,3315.95022700947);

\path [draw=forestgreen4416044, semithick]
(axis cs:57,2797.88144711063)
--(axis cs:57,3004.61763784779);

\path [draw=forestgreen4416044, semithick]
(axis cs:60,2744.15604371294)
--(axis cs:60,3027.32380961875);

\path [draw=crimson2143940, semithick]
(axis cs:0,5859.57295771501)
--(axis cs:0,10985.0366435264);

\path [draw=crimson2143940, semithick]
(axis cs:3,3196.04456764029)
--(axis cs:3,6153.79784612544);

\path [draw=crimson2143940, semithick]
(axis cs:6,2740.11787551273)
--(axis cs:6,4023.34411878597);

\path [draw=crimson2143940, semithick]
(axis cs:9,2304.59889349306)
--(axis cs:9,3500.65423769762);

\path [draw=crimson2143940, semithick]
(axis cs:12,2113.72451473716)
--(axis cs:12,3170.3507610134);

\path [draw=crimson2143940, semithick]
(axis cs:15,2161.47444907919)
--(axis cs:15,3232.92911924918);

\path [draw=crimson2143940, semithick]
(axis cs:18,2013.15035849182)
--(axis cs:18,3110.04125059869);

\path [draw=crimson2143940, semithick]
(axis cs:21,1952.14715062923)
--(axis cs:21,2982.86731591744);

\path [draw=crimson2143940, semithick]
(axis cs:24,1931.3052961825)
--(axis cs:24,2872.6730001635);

\path [draw=crimson2143940, semithick]
(axis cs:27,1870.63936145294)
--(axis cs:27,2997.9085167408);

\path [draw=crimson2143940, semithick]
(axis cs:30,1923.9421503048)
--(axis cs:30,3005.82294804365);

\path [draw=crimson2143940, semithick]
(axis cs:33,1838.95775756318)
--(axis cs:33,2849.65055813548);

\path [draw=crimson2143940, semithick]
(axis cs:36,1870.6301455889)
--(axis cs:36,3120.97833319712);

\path [draw=crimson2143940, semithick]
(axis cs:39,1850.07363534892)
--(axis cs:39,2808.43573727493);

\path [draw=crimson2143940, semithick]
(axis cs:42,1491.02046305185)
--(axis cs:42,3879.8960688791);

\path [draw=crimson2143940, semithick]
(axis cs:45,1838.67762800455)
--(axis cs:45,2949.20604944276);

\path [draw=crimson2143940, semithick]
(axis cs:48,1806.35950910498)
--(axis cs:48,2657.83037223755);

\path [draw=crimson2143940, semithick]
(axis cs:51,1709.35015053629)
--(axis cs:51,2654.56413057926);

\path [draw=crimson2143940, semithick]
(axis cs:54,1759.74874764119)
--(axis cs:54,2618.81135175111);

\path [draw=crimson2143940, semithick]
(axis cs:57,1704.59258340788)
--(axis cs:57,2845.47382689874);

\path [draw=crimson2143940, semithick]
(axis cs:60,1800.35277361407)
--(axis cs:60,2597.91120921587);

\addplot [semithick, steelblue31119180, mark=*, mark size=2, mark options={solid}]
table {%
0 10139.4191443311
3 7837.56780418185
6 6604.6553691942
9 6302.46697107499
12 5654.70608051077
15 4764.56539168208
18 3776.82185676354
21 3152.04081513857
24 2793.45824447206
27 2522.41999762058
30 2496.18215275663
33 2362.98813660762
36 2337.62671742708
39 2244.0271741638
42 2357.57064581821
45 2338.57453502767
48 2237.30394854569
51 2154.53843198308
54 2118.00463567314
57 2121.83004116625
60 2253.27938935117
};
\addplot [semithick, darkorange25512714, mark=triangle, mark size=2, mark options={solid}]
table {%
0 10009.0085648193
3 8733.54797632973
6 7074.14759563696
9 5745.09253299234
12 5375.43549738427
15 5039.94414809935
18 4585.08404474223
21 4132.50122753166
24 3779.75602930986
27 3445.91863684324
30 3249.30647438997
33 3053.90394722467
36 2818.55721655641
39 2705.56677435716
42 2677.26398789037
45 2554.45114605231
48 2482.62517219251
51 2460.43017802404
54 2426.87291770727
57 2386.73102595728
60 2372.50445603565
};
\addplot [semithick, forestgreen4416044, mark=square, mark size=2, mark options={solid}]
table {%
0 9017.98886334417
3 5869.52816956997
6 5239.0351367135
9 4138.98935941994
12 3550.67762778006
15 3603.93596894929
18 3475.69199439526
21 3439.27975942306
24 3176.01157481714
27 3136.20429848045
30 3449.31630964233
33 3182.380511841
36 3006.71253754838
39 3069.5390526079
42 3063.55170673979
45 3050.3191680882
48 3011.64972056737
51 2952.14168772145
54 3064.55800068629
57 2901.24954247921
60 2885.73992666584
};
\addplot [semithick, crimson2143940, mark=diamond, mark size=2, mark options={solid}]
table {%
0 8422.30480062068
3 4674.92120688287
6 3381.73099714935
9 2902.62656559534
12 2642.03763787528
15 2697.20178416418
18 2561.59580454526
21 2467.50723327333
24 2401.989148173
27 2434.27393909687
30 2464.88254917422
33 2344.30415784933
36 2495.80423939301
39 2329.25468631192
42 2685.45826596548
45 2393.94183872366
48 2232.09494067126
51 2181.95714055777
54 2189.28004969615
57 2275.03320515331
60 2199.13199141497
};
\legend{PPO, TRPO, SMC, RB-CSMC}
\end{axis}

\end{tikzpicture}

          \end{figure}
        \end{column}
        \begin{column}{0.5\textwidth}
          \begin{figure}[htbp]
            \centering
            \includegraphics[scale=0.1]{figures/psoc_qr.png}
            \caption{\url{github.com/hanyas/psoc.git}}
          \end{figure}
        \end{column}
      \end{columns}
    \end{frame}

    \section{Bayesian Experimental Design}
    \begin{frame}{Bayesian Experimental Design}
      Let
      \begin{itemize}
        \item $\theta \in \Theta$ be a set of unknown parameters of interest with prior $p(\theta)$.
        \item $\xi \in \Xi$ be a user controllable \emph{design}.
        \item $p(x \mid \xi, \theta)$ be a likelihood function.
      \end{itemize}
      
      \pause
      \begin{definition}
        The \emph{information gain}~\citep[IG,][]{lindley1956measure}, in the parameter $\theta$, on applying the design $\xi$ and observing the outcome $x$ is defined as
        \begin{equation}\label{eq:ig_single_experiment}
          \mathcal{G}(x, \xi) \coloneq \mathbb{H}[p(\theta)] - \mathbb{H}[p(\theta\mid x, \xi)].
        \end{equation}
      \end{definition}

      \pause
      \begin{definition}
        The \emph{expected information gain}~(EIG) is defined as
        \begin{equation}
          \mathcal{I}(\xi) \coloneq \mathbb{E}_{p(x \mid \xi)} \, \mathcal{G}(x, \xi).
        \end{equation}
      \end{definition}
    \end{frame}

    \begin{frame}{Sequential Bayesian Experimental Design}
      Let us define $z_0 \coloneq \{x_0\}$ and $z_{t} \coloneq \{x_t, \xi_{t-1}\}$ for all $t \geq 1$, and denote the outcome-design history up to time $t$ by $z_{0:t} \coloneq \{x_{0:t}, \xi_{0:t-1}\}$. The system dynamics is specified by
      \begin{gather}
        x_0 \sim p(x_0), \\
        \xi_{t-1} \sim \pi_\phi(\xi_{t-1} \mid z_{0:t-1}), \\
        x_t \sim f(x_t \mid x_{t-1}, \xi_{t-1}, \theta),
      \end{gather}
      \pause

      \vspace{0.2cm}
      The joint density of states and designs is hence
      \begin{align}\label{eq:joint_density}
        p_{\phi}(z_{0:T} \mid \theta) &= p(z_{0}) \prod_{t=1}^T p_{\phi}(z_{t} \mid z_{0:t-1}, \theta) \\
        &\coloneq p(x_0) \prod_{t=1}^T f(x_t \mid x_{t-1}, \xi_{t-1}, \theta) \, \pi_\phi(\xi_{t-1} \mid z_{0:t-1})
      \end{align}
    \end{frame}

    \begin{frame}{Sequential Bayesian Experimental Design}
      \begin{itemize}
        \item The EIG for the sequential problem is
          \begin{equation}
            \mathcal{I}(\phi) \coloneq \mathbb{E}_{p_{\phi}(z_{0:T})} \Bigl[ \mathbb{H}\bigl[p(\theta)\bigr] - \mathbb{H}\bigl[p(\theta \mid z_{0:T})\bigr] \Bigr].
          \end{equation}
        \item Our goal is to find optimal policy parameters $\phi^*$ such that
          \begin{equation}
            \phi^* \coloneq \argmax_{\phi \in \Phi} \mathcal{I}(\phi)
          \end{equation}
      \end{itemize}
    \end{frame}

    \begin{frame}{Sequential Bayesian Experimental Design}
      \begin{proposition}
        For models with additive, constant noise in the dynamics, the EIG can be written as
        \begin{equation}\label{eq:eig_constant_noise}
            \mathcal{I}(\phi) \equiv \mathbb{E}_{p_{\phi}(z_{0:T})} \left[ \sum_{t=1}^T r_{t}(z_{0:t}) \right],
        \end{equation}
        where
        \begin{gather}
          r_{t}(z_{0:t}) = - \log \int p(\theta \mid z_{0:t-1}) \, f(x_t \mid x_{t-1}, \xi_{t-1}, \theta) \, \dd \theta, \\
          p_\phi(z_{0:T}) = \int p_\phi(z_{0:T} \mid \theta) \, p(\theta) \, \dd \theta.
        \end{gather}
      \end{proposition}
    \end{frame}

  \section{The dual inference problem}

    \begin{frame}{BED as probabilistic inference}
      Let us introduce a potential function
      \begin{equation}\label{eq:potential-function}
        g_{t}(z_{0:t}) \coloneq \exp \Big\{ \eta \, r_{t}(z_{0:t}) \Big\},
      \end{equation}
      with $\eta \in \mathbb{R}_{>0}$.

      \vspace{0.3cm}
      We then define a non-Markovian state-space model characterized by the following joint density
      \begin{equation}\label{eq:pathwise_smoothing_trajectory}
        \Gamma_\phi(z_{0:T}) = \frac{1}{Z(\phi)} \, p(z_0) \prod_{t=1}^T p_\phi(z_t \mid z_{0:t-1}) \, g_t(z_{0:t}),
      \end{equation}
      where the normalization constant is
      \begin{equation}
        Z(\phi) = \int g_{1:T}(z_{0:T}) \, p_\phi(z_{0:T}) \, \dd z_{0:T}.
      \end{equation}
    \end{frame}

    \begin{frame}{BED as probabilistic inference}
      Some observations:
      \begin{itemize}
        \item $Z(\phi)$ is analogous to the marginal likelihood $\ell(\phi) = p_\phi(y_{0:T} = 1)$.
        \item $\Gamma_\phi(z_{0:T})$ is analogous to the smoothing distribution $p_\phi(z_{0:T} \mid y_{0:T} = 1)$.
        \item Maximizing the (log) marginal likelihood targets the risk-sensitive objective:
          \begin{align}
            \argmax_{\phi \in \Phi} \log Z(\phi) &= \argmax_{\phi \in \Phi} \frac{1}{\eta} \log \mathbb{E}_{p_\phi(z_{0:T})} \left[\exp \left\{\eta \sum_{t=1}^T r_t(z_{0:t})\right\}\right] \\
            &= \mathbb{E} \left[\sum_{t=1}^T r_t(z_{0:t})\right] - \frac{\eta}{2} \mathbb{V} \left[ \sum_{t=1}^T r_t(z_{0:t}) \right] + \mathcal{O}(\eta^2).
          \end{align}
      \end{itemize}
    \end{frame}

    \begin{frame}{IBIS}
      
    \end{frame}

  % Bibliography
  \begin{frame}[t, allowframebreaks]{References}
    \bibliography{references}
  \end{frame}

\end{document}
